% This file was converted to LaTeX by Writer2LaTeX ver. 1.0.2
% see http://writer2latex.sourceforge.net for more info
\documentclass[letterpaper]{article}
\usepackage[latin1]{inputenc}
\usepackage[T1]{fontenc}
\usepackage[english]{babel}
\usepackage{amsmath}
\usepackage{amssymb,amsfonts,textcomp}
\usepackage{color}
\usepackage{array}
\usepackage{hhline}
\usepackage{hyperref}
\hypersetup{pdftex, colorlinks=true, linkcolor=blue, citecolor=blue, filecolor=blue, urlcolor=blue, pdftitle=, pdfauthor=, pdfsubject=, pdfkeywords=}
% Page layout (geometry)
\setlength\voffset{-1in}
\setlength\hoffset{-1in}
\setlength\topmargin{2cm}
\setlength\oddsidemargin{2cm}
\setlength\textheight{23.94cm}
\setlength\textwidth{17.59cm}
\setlength\footskip{0.0cm}
\setlength\headheight{0cm}
\setlength\headsep{0cm}
% Footnote rule
\setlength{\skip\footins}{0.119cm}
\renewcommand\footnoterule{\vspace*{-0.018cm}\setlength\leftskip{0pt}\setlength\rightskip{0pt plus 1fil}\noindent\textcolor{black}{\rule{0.25\columnwidth}{0.018cm}}\vspace*{0.101cm}}
% Pages styles
\makeatletter
\newcommand\ps@Standard{
  \renewcommand\@oddhead{}
  \renewcommand\@evenhead{}
  \renewcommand\@oddfoot{}
  \renewcommand\@evenfoot{}
  \renewcommand\thepage{\arabic{page}}
}
\makeatother
\pagestyle{Standard}
\title{}
\author{}
\date{2010-05-29}
\begin{document}
{\bfseries
Formulas used in 1D SPH Code:}


\bigskip

The equations employed are Euler Equation (without physical viscosity)


\bigskip

SPH Form of \textbf{continuity equation}:


\bigskip

For Sum Density mode:

\ \ \ \  $\rho _{a}=\sum {m_{b}W_{b}}$ which corresponds to (2.9)


\bigskip

For integrating density mode:

\ \ \ \  $\frac{d\rho _{a}}{\mathit{dt}}=\sum
{m_{b}v_{\mathit{ab}}\nabla _{a}W_{\mathit{ab}}}$ which corresponds to
(2.18)


\bigskip


\bigskip

SPH Form of \textbf{Momentum equation:}


\bigskip

as this is shock simulation a formulation including an artificial
viscosity is used:


\bigskip

\ \ \ \  $\frac{dv_{a}}{\mathit{dt}}=-\sum {m_{b}(\frac{P_{a}}{\rho
_{a}^{2}}+\frac{P_{b}}{\rho _{b}^{2}}+\Pi _{\mathit{ab}})\nabla
_{a}W_{\mathit{ab}}}$(8.2)


\bigskip

\ \ \ \  $\Pi _{\mathit{ab}}=-\nu
(\frac{v_{\mathit{ab}}r_{\mathit{ab}}}{r_{\mathit{ab}}^{2}+\epsilon
h_{\mathit{ab}}^{2}})$(8.3)


\bigskip

\ \ \ \  $\nu =\frac{h_{\mathit{ab}}}{\rho _{\mathit{ab}}}(\alpha
c_{\mathit{ab}}-\beta
\frac{h_{\mathit{ab}}v_{\mathit{ab}}r_{\mathit{ab}}}{r_{\mathit{ab}}^{2}+\epsilon
h_{\mathit{ab}}^{2}})$(8.10)


\bigskip


\bigskip

SPH Form of \textbf{Energy Equation}


\bigskip

including the artificial viscosity term (to be consistent, we have to
take into account the artificial viscosity in the energy equation as
well. For viscosity leads to a heating of the fluid)


\bigskip

\begin{equation*}
\frac{de_{a}}{\mathit{dt}}=-\mathit{}\frac{1}{2}\sum
{m_{b}(\frac{P_{a}}{\rho _{a}^{2}}+\frac{P_{b}}{\rho _{b}^{2}}+\Pi
_{\mathit{ab}})v_{\mathit{ab}}\nabla _{a}W_{\mathit{ab}}}
\end{equation*}

\bigskip

This equation in the above form can not directly be found in the
Monaghan 2005 paper, but as (4.78) in the book smoothed particle
hydrodynamics (Liu).
\end{document}
